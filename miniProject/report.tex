\documentclass[12pt]{article}
\usepackage[utf8]{vietnam}
\usepackage{tikz}
\usepackage{enumerate}
% \usepackage{mathptmx}
\usepackage{geometry}
\usepackage{ragged2e}
 \geometry{
 a4paper,
% total={170mm,257mm},
 left=15mm,
 top=15mm,
 right=15mm,
 }
 \thispagestyle{empty}
 \usepackage{enumitem}
 \usepackage{array}
 % \usepackage[16pt]{moresize}
\begin{document}
\setlist[description]{font=\normalfont\itshape\space}
    % \begin{minipage}[t][][b]{0.45\textwidth}
    %     \centering 
    %     TRƯỜNG ĐẠI HỌC BÁCH KHOA HÀ NỘI\\
    %     \textbf{VIỆN CÔNG NGHỆ THÔNG TIN VÀ TRUYỀN THÔNG} \\
    %     \begin{tikzpicture}
    %         \draw(0,0) -- (4,0);
    %     \end{tikzpicture}
    % \end{minipage}
    % \begin{minipage}[t][][b]{0.55\textwidth}
        \centering 
        \textbf{HANOI UNIVERSITY OF SCIENCE AND TECHNOLOGY}\\
        \textbf{School of Information and Communication Technology} \\
        \begin{tikzpicture}
            \draw(0,0) -- (4,0);
        \end{tikzpicture}
    % \end{minipage}
    % \vspace{20pt}
    \begin{flushright}
        \textit{Hanoi, May 25th 2020}
    \end{flushright}
    {\center \textbf{COMPUTER ARCHITECTURE EXPERIMENTAL REPORT} \vspace{15pt} \\
        %\textit{Kính gửi: } FAMILY TECHNOLOGY COMPANY LIMITED
        %\par
    }
    \begin{description}%[topsep=0pt, partopsep=0pt]
        \setlength\itemsep{0pt}
        \item[Full name:] Nguyễn Văn Lực
        %\item[Mảng:] Recommend System 
		\item[Student ID:] 20176812
		\item[Topic:] Create a program to:
		\begin{itemize}
		\item[-] Input an array of integers from the keyboard.
		\item[-] Find the maximum element of the array.
		\item[-] Calculate the number of elements in the range of (m, M). Range m, M are inputted from the keyboard.
		\end{itemize}
        % \item[Lớp, Khóa:] TT CNTT ICT 01 - K62
%        \item thoại:] 037 555 0576 
%        \item[Thời gian thực tập:] Từ 23 tháng 9 năm 2019 
%        \item[Email:] nguyenlucitbk@gmail.com 
    \end{description}
\begin{enumerate}[label=\textbf{\Roman*}]
        \item \textbf{Procedure:} \\
        \begin{enumerate}
        	\item Prompt to input the number of element and value of elements in array.
        	\item Initialize $max$ value to be the first number of the array.
        	\item For each element inserted, check if it is greater than current maximum value, and assign new maximum if it is greater.
        	\item Print out the maximum value when all n elements are inserted.
        	\item Prompt to input 2 value, $m$ and $M$.
        	\item Check if $m < M$. If not, quit the program since $(m,M)$ is not a valid range. If yes, initialize a variable $count = 0$ to count the number of element in array satisfied the condition.
        	\item For each $value$ of element in array, check if $m < value$ and $value < M$. If satisfied, increase the variable $count$ by 1.
        	\item When reach the last element, print the value of $count$ and quit the program.
        \end{enumerate}
        \item \textbf{The meaning of used registers}
        \begin{description}
        	\item{ $\$s0$} : Store the number of elements ($n$) in array, this will not be changed.
        	\item $\$s1$ : Store the maximum element of the array, this will be changed while searching for maximum element.
        	\item $\$s2$ : Store the pointer to the last element of the array, this register will be decreased from the register $\$sp$ continuously.
        	\item $\$s3$ : Store the value of $m$.
        	\item $\$s4$ : Store the value of $M$.
        	\item $\$s5$ : Store the value of the variable $count$.
        	\item $\$t0$ : Running index, from 0 to $n$ ($\$s0$)
        	\item $\$t1$ : Store the index to print, this always equals to $\$s1 + 1$, used to print to the user the order of element needed to be inserted.
        	\item $\$t2$ : Temporarily store the value of the $\$t1$-th element inserted above before saving it to stack.
        	\item $\$t3$ : First, check whether the current maximum value is smaller than the value of new element inserted above. Second, re-use it to check if $m < M$, if $m < value$, and if $value < M$ where $value$ is value of element loaded from the array to check if it is in range $(m, M)$.
        	\item $\$t4$ : Running address, from $\$s2$ stored above to $\$sp$ to get the value of element in array.
        	\item $\$t5$ : Temporarily store the value of element loaded by $\$t4$. 
        \end{description}
%        \end{itemize}
        \item \textbf{The meaning of used sub-program}\\
        Here I used programs defined in the library $utils.asm$ as following:
        \begin{description}
        	\item [PromptInt :] Used to print the string whose address is loaded in the register $\$a0$, prompt to get new integer value inserted from keyboard, which is then stored in the register $\$v0$.
        	\item[PrintString :] Used to print the string whose address is loaded in the register $\$a0$.
        	\item[PrintInt :] Used to print the string whose address is loaded in the register $\$a0$ and an integer whose value is loaded in the register $\$a1$.
        	\item[Exit :] Used to quit the program.
        \end{description}
\end{enumerate}


\end{document}